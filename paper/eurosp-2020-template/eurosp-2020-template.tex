\documentclass[compsoc,conference,a4paper,10pt,times]{IEEEtran}
% \IEEEoverridecommandlockouts
% The preceding line is only needed to identify funding in the first footnote. If that is unneeded, please comment it out.
\usepackage{cite}
\usepackage{amsmath,amssymb,amsfonts}
\usepackage{algorithmic}
\usepackage{graphicx}
\usepackage{textcomp}
\usepackage{bmpsize}
\usepackage{xcolor}
\usepackage{lipsum}
\usepackage[colorlinks=true,urlcolor=black]{hyperref}
\def\BibTeX{{\rm B\kern-.05em{\sc i\kern-.025em b}\kern-.08em
    T\kern-.1667em\lower.7ex\hbox{E}\kern-.125emX}}
    
\usepackage{minted}    

\usepackage[inline]{enumitem}
% New commands

\begin{document}

\title{Towards Certified Program Obfuscation}

\author{\IEEEauthorblockN{Weiyun Lu}
\IEEEauthorblockA{\textit{School of Electrical Engineering and Computer Science} \\
\textit{University of Ottawa}\\
Ottawa, Canada \\
WLU058@uottawa.ca}
\and
\IEEEauthorblockN{Bahman Sistany}
\IEEEauthorblockA{\textit{Cloakware Research} \\
\textit{Irdeto Canada}\\
Ottawa, Canada \\
bahman.sistany@irdeto.com}
\and
\IEEEauthorblockN{Amy Felty}
\IEEEauthorblockA{\textit{School of Electrical Engineering and Computer Science} \\
\textit{University of Ottawa}\\
Ottawa, Canada \\
afelty@uottawa.ca}
\and
\IEEEauthorblockN{Philip Scott}
\IEEEauthorblockA{\textit{School of Electrical Engineering and Computer Science} \\
\textit{University of Ottawa}\\
Ottawa, Canada \\
philip.scott@uottawa.ca}
}

\maketitle

\begin{abstract}
%We expect our systems including software systems to function ``correctly''. By ``correctly'', we mean that a system will behave according to explicit and/or implicit expectations. Typically, extensive testing is done to increase the confidence in correct functionality of a piece of software but positive test results are not a proof of correctness. In \emph{High Assurance} systems, formal verification based methods are used. Behaviour of interest such as functionality, safety and security may be expressed via some sort of formal specification language.

\par How can one perform code transformations such as obfuscating transformations or optimizing transformations on code that is assumed to be correct with respect to certain specified behaviour? Will the transformed code preserve the specified behaviour as one expects? 

\par To achieve the highest levels of assurance that the transformations have maintained correctness of the code, is to prove the two versions of the program (before and after the transformation) is equivalent. Total equivalency between the two versions of a program certainly implies the correctness of any specified properties of interest but what if we directly try to show validity of these properties on the transformed program? 

\par In this thesis, we lay the foundation to study and reason about code obfuscating transformations and show how the preservation of certain behaviours may be ``certified''. To this end, we apply techniques of formal specification and verification, by using the Coq Proof Assistant and IMP (a simple imperative language within it), to formulate what it means for a program's semantics to be preserved by an obfuscating transformation, and give formal machine-checked proofs that these properties hold. 

\par We describe our work on opaque predicates, a simple control flow obfuscation, and elements of control flow flattening transformation, a more complex control flow obfuscation.  Along the way, we employ Hoare logic as our foundational specification language, as well as augment the IMP language with Switch statements.  We also define a lower-level flowchart language to wrap around IMP for modelling certain flattening transformations, treating blocks of codes as objects in their own right.

\end{abstract}

\begin{IEEEkeywords}
obfuscation, verification, security, correctness, Coq, proof
\end{IEEEkeywords}

\section{Introduction}
\subsection{Background and Motivation}
We expect our systems including software systems to function ``correctly''. By ``correctly'', we mean that a system will behave according to explicit and/or implicit expectations or said another way to its written and/or unwritten specifications. Typically, extensive testing is done to increase the confidence in correct functionality of a piece of software but alas testing is known to be based on inductive reasoning where more tests passing can only increase the likelihood of correctness, so positive testing results are not a proof of correctness. In systems where more assurance of correctness is required various types of deductive reasoning is used. These are formal verification methods based on theoretical foundations rooted in logic. It is important to note that, formal verification transfers the problem of confidence in program correctness to the problem of confidence in specification correctness, so it is not a silver bullet however since specifications are often smaller and less complex to express, we have successfully reduced the trusted computing base (TCB) and increased our chances of achieving correctness.

Formal verification based methods, used to show a (software) system behaves as its specification says, typically employ a specification language based on the familiar ``assertions''. A specification is typically expressed in some variation of first order logic and the verification system will deductively try to prove the assertions correct or signal that they don't hold. This is a rather elaborate process where assertions (general propositional statements about program fragments that are expected to hold) are used to generate verification conditions (VC), logic formulas, that are then fed into a satisfiability modulo theories (SMT) solver, either behind the scenes in a verification backend or in more visible to the verification expert. VC generation for program verification goes back to at least Hoare's triples, Eiffel style contracts and proof-carrying-code (PCC) of Necula\cite{b7}.

How can one perform code transformations such as obfuscating transformations or optimizing transformations on code that is assumed to be correct with respect to certain specified behaviour (expressed in some assertion language) while preserving the correctness of the specified behaviour? 

To achieve the highest levels of assurance that the transformations have maintained correctness of the code, is to prove the two versions of the program (before and after the transformation) is equivalent. Proving equivalency is doable in certain cases but in general is still extremely hard to do for general programs. Despite the difficulty, there are now formally verified compilers such as CompCert. However, the problem with verifying realistic systems such as a compiler is scale. CompCert verification of semantic equivalence between C and generated assembly took several man years to complete. The cost of using formal verification for mere mortals (on realistic systems) is still high. 

Total equivalency between the two versions of a program certainly implies the correctness of any program properties of interest but what if we directly try to show validity of these properties? What if we limit ourselves to only proving properties of interest in the ``before'' version of a program are maintained in the ``after'' version of the program (after a transformation is applied)? 

Certain simple transformations simply don't invalidate expressed properties about the ``before'' version versus the ``after'' version. Below the program snippet in listing [\ref{lst:before}] asserts that $y > 2$ which we can verify visually to be true. In the snippet in listing [\ref{lst:after}] we use a simple obfuscating transformation called variable splitting where we have split the variable $x$ into two other variables $x1$ and $x2$ and we see that (visually) the assertion $y > 2$ still holds.


\begin{listing}
\caption{Original Code}
\label{lst:before}
\begin{minted}
[
frame=single,
framesep=2mm,
baselinestretch=1.2,
fontsize=\footnotesize
] {c}
x = 2; y = 5;
y = x + y;
assert(y > 2);
\end{minted}
\end{listing}

\begin{listing}
\caption{Obfuscated Code}
\label{lst:after}
\begin{minted}
[
frame=single,
framesep=2mm,
baselinestretch=1.2,
fontsize=\footnotesize
] {c}
x1 = 1; x2 = 1; y = 5;
y = x1 + y; y = x2 + y;
assert(y > 2);
\end{minted}
\end{listing}

In general, though, most transformations, whether optimizations or obfuscations but specially obfuscations, invalidate assertions that hold true about the ``before'' version. Obfuscation is especially troublesome because the goal of obfuscation is to hide the functionality of the code from prying eyes while maintaining the functionality of the ``before'' program. ``Prying eyes'' could as easily be the same as some kind of static analysis tool where an attacker is trying to determine certain facts about the code and obfuscation is trying to make this difficult. The program in listing [\ref{lst:beforeopaque}] is correct with respect to the assertion that is expressed (e.g. $z == 30$) as is evident by simple inspection of the code. The program snippet in listing [\ref{lst:afteropaque}] is the ``after'' program where a non-linear opaque predicate transformation has been applied to hide the fact that at program's end, value of $z$ is in fact 30. We can see that the transformation makes it a bit harder to see that the assertion still holds but knowing the fact that $\forall x \in \mathbb{Z}, ((x^2 + x)\bmod 2) == 0$ , we can deduce that the assertion does hold and the value of $z$ is in fact still 30. 

This paper describes steps towards implementing a framework in the Coq Proof Assistant and based on IMP\cite{SFV2}, a simple imperative language. to study obfuscating transformations, their impact on programs and how specified behaviour may be preserved beyond the transformations. A of number of initial goals and principles drove the direction of this research: 
\begin{enumerate*}
  \item Not reinventing the wheel: start out with IMP a familiar small imperative language implemented in Coq and use its accompanying formalized semantics in \cite{SFV2}.
  \item Accessibility to as wide an audience as possible: an obvious option was to use CompCert and Clight as \cite{Blazy2} have done. We would have started with lots of proofs and formalisms for free (already done by the CompCert team) however the significant learning curve associated with learning CompCert infrastructure seemed prohibitive. We deemed IMP and Coq much more accessible.\label{goal2}
  \item Extendibility of the framework: Following the lead of \cite{SFV2} where a number of extensions to IMP are easily added and studied, we wanted the ability to build our obfuscation infrastructure incrementally on top of IMP.\label{goal3}
  \end{enumerate*} 
  
Keeping these research goals in mind the contributions of this paper are the following:

\begin{itemize}
    \item We consider different formulations of what it means for a transformation to be semantics-preserving, including complete state equivalence as well as Hoare logic equivalence. In this particular setting, the latter is a novel approach, and we give examples of its use with opaque predicate transformations. In addition, use of Hoare logic in this context leads to establishing our main approach to ``certifying obfuscating transformations'': our obfuscating transformations will be ``decorated''  \`a la  Pierce's \cite{SFV2} with additional assertions and their proofs.\label{itm:1}
    \item We give clear and detailed explanations of the proofs and tactics in Coq, which, to the best of our knowledge, the existing literature does not, thus providing an accessible explanation of not just obfuscation techniques, but also in tandem with its formalization and verification inside Coq. This follows from research goal \ref{goal2}.\label{itm:2}
    \item We begin with a minimal imperative programming language inside Coq for reasoning about programs and their transformations, and then augment it as needed for control flow flattening algorithms, first by augmenting its syntax and semantics with switch statements, and then by defining a lower-level flowchart language that wraps around blocks of code in order to model real-world intermediate languages used in obfuscation tools. This follows our research goals \ref{goal2} and \ref{goal3}. \label{itm:3}
\end{itemize}


%This example is a good preview of a key part of our research, where we may have to supply additional facts to show that our transformations indeed maintain prior program properties expressed as specifications. The assertions will capture properties or conditions that need to hold true for the transformed code as evidence the transformations have not invalidated the specifications of the original code. Think of the assertions as a kind of a key that can undo the transformations to show that the desired behaviour (as encoded in the specifications) are still valid. For some of the more sophisticated transformations we will need to help the verification to discharge (i.e. to solve) our assertions. We will accomplish this by providing ``hints'' along with the corresponding assertions. In formal annotation language settings, assertions are specified using the familiar `assert' facility while the hints are expressed using the 'assume' facility. 
% Bahman: need to tie Hoare logic as a starting point to eventually having a small assertion language where we supply additional facts as `assume' statements. Of course this assertion language is part of our future work and not yet implemented.


%The idea is to accompany the transformations with invariants or assertions that will help the verification process discharging the transformed programsin some assertion language starting with using Hoare logic. The invariants will capture properties or conditions that need to hold true for the transformed code as evidence the transformations have not invalidated the specifications of the code. Think of the invariants as a kind of a key that can undo the transformations to show that the desired behaviour (as encoded in the specifications) are still valid. 

\begin{listing}
\caption{Original Code}
\label{lst:beforeopaque}
\begin{minted}
[
frame=single,
framesep=2mm,
baselinestretch=1.2,
fontsize=\footnotesize
] {c}
int main (int argc, char *argv[])
{
  unsigned int x = 10;
  unsigned int y = 20;
  unsigned int z = 0;

  z = x + y;
  assert(z == 30);
  return 0;
}
\end{minted}
\end{listing}

\begin{listing}
\caption{Obfuscated Code}
\label{lst:afteropaque}
\begin{minted}
[
frame=single,
framesep=2mm,
baselinestretch=1.2,
fontsize=\footnotesize
] {c}
int main (int argc, char *argv[])
{
  unsigned int x = 10;
  unsigned int y = 20;
  unsigned int z = 0;

  unsigned int a = ((unsigned int)argc);
  unsigned int w = a * a;

  w = a + w;
  w = w % 2;
  if (w == 0)
  {
    z = x + y;
  }
  else
  {
    z = y - x;
  }
  assert(z == 30);
  return 0;
}
\end{minted}
\end{listing}


\section{Opaque predicates in IMP/Coq}

We now enter the main topic of the thesis proper, formalizing and certifying the opaque predicate transformation mentioned earlier. 

\subsection{Opaque predicate obfuscation}\label{subsec: opaquedefn}
	An \emph{opaque predicate} \cite{CoNa} is a predicate\footnote{This could be any statement in a program that could evaluate to true or false, but we will only be concerned with arithmetic formulas in this thesis.} that always evaluates to either \emph{true} or \emph{false} and the truth-value of which is known to the programmer writing the code. The code under the \emph{false} branch is never evaluated at runtime so \emph{opaque predicates} incur no runtime performance.

Of course, the absolutely most basic opaque predicates are just the boolean constants \emph{true} and \emph{false} themselves, but these are not very useful in practice because it is immediately obvious what is happening in the program, and neither the simplest of humans nor tools will be fooled. For a more advanced treatment of opaque predicates and how they may be detected see \cite{Prada}.


An  \emph{opaque predicate} transformation takes as inputs a program to be obfuscated, $c_1$, an opaque predicate $P$, and a dummy program\footnote{It's not known to an attacker, a priori, that it's a dummy program.  In practice, $c_2$ should be constructed so that it is not obvious; e.g.\ $c_2$ should not be simply an empty program, but should be complicated enough that it looks like it could feasibly be intended to be executed.} $c_2$, and returns the program
\begin{verbatim}
    IFB (P x) THEN c1 ELSE c2 FI.
\end{verbatim}




\section{Related works}
There have been three papers, in all of which Sandrine Blazy (Universit\'{e} de Rennes 1) appears as a coauthor, that study code obfuscation in Coq.

\subsection*{Towards a formally verified obfuscating compiler}
\par \emph{Towards a formally verified obfuscating compiler} \cite{Blazy1} also uses IMP as the language for obfuscation, but studies data obfuscation techniques, as opposed to the control obfuscation techniques which opaque predicates and control flow flattening fall under \cite{CollbergTax}.
\par The first particular transformation studied herein is obfuscating integer constants, wherein all integer values are substituted by different ones in a distorted semantics using an obfuscating function $O: \mathbb{N} \to \mathbb{N}$.  The other discussed is variable encoding, which changes the names of variables.  A real-life application of this could be, for instance, to change a descriptive variable name like $account\_balance$ to a string of gibberish.
\par This is an inherently different class of techniques from the ones studied in the present work, and one can make a simple combinatorial argument that putting them together in the same obfuscation transformation would generate a synergistic effect on making a program more difficult to analyze.

\subsection*{Formal verification of control-flow graph flattening}
\par \emph{Formal verification of control-flow graph flattening} \cite{Blazy2} also studies control flow flattening, but the authors use the Clight language of CompCert (the formally verified C compiler in Coq, discussed in Chapter % Bahman: \ref{two}).  Should refer to a section: Background --- formal verification}\label{two}
Clight is the first intermediate language in the CompCert compiler, and the strategy used was to prove the correctness of the obfuscation strictly there, from which CompCert's own proofs of semantic preservation give the correctness of the rest of the compilation process ``for free".
\par On the one hand, this makes the work less elementary and less accessible, as it works with a nontrivial subset of the real C language, but on the other it is clear evidence that formal verification of obfuscation techniques need not be restricted to a small language like IMP (which would never be used in real software development), and other real-world practicalities considered in this paper include simulation techniques and analysis of running time.
\par This work also discusses some techniques for combining obfuscation techniques, such as splitting a switching variable into two different variables that are updated at different points of a program, as well as randomly encoding the values of the switch cases so that they are not just consecutive numbers beginning with 1.  These are necessary considerations, since we need to think one level higher about our attackers, and obfuscate the fact that we are obfuscating particular parts of our code with CFG flattening in the first place!
\par In comparing this work to ours, the present author believes there is merit both in the IMP and the CompCert routes.  In the former, the language used is of minimal complexity, which allows not only for specifications and proofs of transformations to be developed quicker without being bogged down in unnecessarily complicated features of the underlying language, but is also better suited for pedagogical purposes.  IMP is also Turing complete, so from a theoretical point of view there is no loss of generality in proofs made using it --- they can always be adapted to CompCert later.  But on the other hand, CompCert is, of course, closer to languages that would be of interest to real-world software development and so more practical in that sense.
\par The authors ran into a similar issue as in the present work of needing to separate switching variables from those in the program to be transformed, but their solution was different --- they instead use a function to parse the program to be transformed and generate a fresh variable which doesn't appear there to be used for the transformation.  From a practical point of view, this is perhaps more natural, and in line with how a real obfuscating tool would function --- generating new variables rather than demand that a certain specifically named variable doesn't exist in the source program.  Theoretically, though, these are equivalent, since any program can contain only finitely many variable names, and there are an infinite number to choose from.

\subsection*{Formal verification of a program obfuscation based on mixed boolean-arithmetic expressions}
\par \emph{Formal verification of a program obfuscation based on mixed boolean-arithmetic expressions} \cite{Blazy3} mixed boolean-arithmetic continues to work in Clight, which studies obfuscations that involve mixing arithmetic operators and bitwise boolean operators.  This is another data obfuscation which appears frequently in real-world binary code, but as it is based on features wildly beyond the capabilities of IMP, a detailed discussion is beyond the scope of the present work.



\begin{thebibliography}{00}
\bibitem{Blazy1} Sandrine Blazy and Roberto Giacobazzi. Towards a formally verified obfuscating compiler. In 2nd ACM SIGPLAN Software security and protection
workshop, SPP 2012, St. Petersburg, FL, USA, 2012. HAL (https://hal.archives-ouvertes.fr/).
\bibitem{Blazy2} Sandrine Blazy and Alix Trieu. Formal verification of control-fow graph flattening. In Proceedings of the 5th ACM SIGPLAN Conference on Certified
Programs and Proofs, CPP 2016, pages 176-187, New York, NY, USA, 2016. ACM.

\bibitem{Blazy3} Sandrine Blazy and Remi Hutin. Formal verification of a program obfuscation based on mixed boolean-arithmetic expressions. In Proceedings of the 8th ACM SIGPLAN International Conference on Certified Programs and Proofs, CPP 2019, Cascais, Portugal, January 14-15, 2019, pages 196-208, 2019.

\bibitem{b2} 	Xavier Leroy:
Formal verification of a realistic compiler. Commun. ACM 52(7): 107-115 (2009)
\bibitem{b3} I. S. Jacobs and C. P. Bean, ``Fine particles, thin films and exchange anisotropy,'' in Magnetism, vol. III, G. T. Rado and H. Suhl, Eds. New York: Academic, 1963, pp. 271--350.
\bibitem{b4} K. Elissa, ``Title of paper if known,'' unpublished.
\bibitem{b5} R. Nicole, ``Title of paper with only first word capitalized,'' J. Name Stand. Abbrev., in press.
\bibitem{b6} Y. Yorozu, M. Hirano, K. Oka, and Y. Tagawa, ``Electron spectroscopy studies on magneto-optical media and plastic substrate interface,'' IEEE Transl. J. Magn. Japan, vol. 2, pp. 740--741, August 1987 [Digests 9th Annual Conf. Magnetics Japan, p. 301, 1982].
\bibitem{b7} George C. Necula: Proof-Carrying Code. POPL 1997: 106-119. 

% New items
\bibitem{CollbergTax} C. Collberg, C. Thomborson, and D. Low.
A Taxonomy of Obfuscating Transformations.
Technical Report 148, Department of Computer Science, University of Auckland, July 1997.
\bibitem{SFV2} Benjamin C. Pierce, Arthur Azevedo de Amorim, Chris Casinghino, Marco Gaboardi, Michael Greenberg, Cat? alin Hri?cu, Vilhelm Sj�berg, Andrew ?
Tolmach, and Brent Yorgey. Programming Language Foundations. Software Foundations series, volume 2. Electronic textbook, May 2018.
\bibitem{CoNa} 	Christian S. Collberg, Jasvir Nagra:
Surreptitious Software - Obfuscation, Watermarking, and Tamperproofing for Software Protection. Addison-Wesley Software Security Series, Addison-Wesley 2010, ISBN 978-0-321-54925-9, pp. I-XXVII, 1-748
\bibitem{Prada} Mila Dalla Preda, Matias Madou, Koen De Bosschere, Roberto Giacobazzi:
Opaque Predicates Detection by Abstract Interpretation. AMAST 2006: 81-95
\end{thebibliography}

\end{document}
